%% 
%% Copyright 2007-2020 Elsevier Ltd
%% 
%% This file is part of the 'Elsarticle Bundle'.
%% ---------------------------------------------
%% 
%% It may be distributed under the conditions of the LaTeX Project Public
%% License, either version 1.2 of this license or (at your option) any
%% later version.  The latest version of this license is in
%%    http://www.latex-project.org/lppl.txt
%% and version 1.2 or later is part of all distributions of LaTeX
%% version 1999/12/01 or later.
%% 
%% The list of all files belonging to the 'Elsarticle Bundle' is
%% given in the file `manifest.txt'.
%% 
%% Template article for Elsevier's document class `elsarticle'
%% with harvard style bibliographic references

\documentclass[preprint,12pt]{elsarticle}
%\documentclass[final,5p,times,twocolumn]{elsarticle} 
% Adapt figure at about line 887 when documentclass is changed

% Suppress display of footer ``Submitted to ...'', see https://tex.stackexchange.com/questions/35712/modify-footer-used-by-elsarticle-cls
\makeatletter
\def\ps@pprintTitle{%
 \let\@oddhead\@empty
 \let\@evenhead\@empty
 \def\@oddfoot{\centerline{\thepage}}%
 \let\@evenfoot\@oddfoot}
\makeatother

%% Use the option review to obtain double line spacing
%% \documentclass[authoryear,preprint,review,12pt]{elsarticle}

%% Use the options 1p,twocolumn; 3p; 3p,twocolumn; 5p; or 5p,twocolumn
%% for a journal layout:
%% \documentclass[final,1p,times,authoryear]{elsarticle}
%% \documentclass[final,1p,times,twocolumn,authoryear]{elsarticle}
%% \documentclass[final,3p,times,authoryear]{elsarticle}
%% \documentclass[final,3p,times,twocolumn,authoryear]{elsarticle}
%% \documentclass[final,5p,times,authoryear]{elsarticle}
%% \documentclass[final,5p,times,twocolumn,authoryear]{elsarticle}

%% For including figures, graphicx.sty has been loaded in
%% elsarticle.cls. If you prefer to use the old commands
%% please give \usepackage{epsfig}

%% The amssymb package provides various useful mathematical symbols
\usepackage{amssymb}
%% The amsthm package provides extended theorem environments
%% \usepackage{amsthm}

%% The lineno packages adds line numbers. Start line numbering with
%% \begin{linenumbers}, end it with \end{linenumbers}. Or switch it on
%% for the whole article with \linenumbers.
%% \usepackage{lineno}

% To be able to enter URL with breaks:
\usepackage[hyphens]{url}
\usepackage{hyperref}
\usepackage{framed}

\sloppy % See anser in https://tex.stackexchange.com/questions/9107/how-can-i-make-my-text-never-go-over-the-right-margin-by-always-hyphenating-or-b  -  This prevents that SAP2Moose or Moose2Model overlap sometimes the right border.


\newdefinition{definition}{Definition}


\journal{}

\begin{document}

\begin{frontmatter}

%% Title, authors and addresses

%% use the tnoteref command within \title for footnotes;
%% use the tnotetext command for theassociated footnote;
%% use the fnref command within \author or \affiliation for footnotes;
%% use the fntext command for theassociated footnote;
%% use the corref command within \author for corresponding author footnotes;
%% use the cortext command for theassociated footnote;
%% use the ead command for the email address,
%% and the form \ead[url] for the home page:
%% \title{Title\tnoteref{label1}}
%% \tnotetext[label1]{}
%% \author{Name\corref{cor1}\fnref{label2}}
%% \ead{email address}
%% \ead[url]{home page}
%% \fntext[label2]{}
%% \cortext[cor1]{}
%% \affiliation{organization={},
%%            addressline={}, 
%%            city={},
%%            postcode={}, 
%%            state={},
%%            country={}}
%% \fntext[label3]{}

\title{File to SOMIX}

%% use optional labels to link authors explicitly to addresses:
%% \author[label1,label2]{}
%% \affiliation[label1]{organization={},
%%             addressline={},
%%             city={},
%%             postcode={},
%%             state={},
%%             country={}}
%%
%% \affiliation[label2]{organization={},
%%             addressline={},
%%             city={},
%%             postcode={},
%%             state={},
%%             country={}}

\author[1]{Rainer Wolfgang Winkler\corref{cor1}%
%\fnref{fn1}
}
\ead{rainer.winkler@cubeserv.com}
\address[1]{CubeServ GmbH, Am Prime-Parc 4, 65479 Raunheim, Germany}

\cortext[cor1]{Corresponding author}

%\author{Rainer Wolfgang Winkler}
%\adress{Test}
%\affiliation{organization={CubeServ GmbH},%Department and Organization
%            addressline={Am Prime-Parc 4}, 
%            city={Raunheim},
%            postcode={65479}, 
%%            state={},
%            country={Germany}}

\begin{abstract}
One of many possible ways to extract a SOMIX model from files with JavaScript coding is specified.

\end{abstract}


\begin{keyword}
% Up to 6 keywords in british spelling. Avoid general and plural terms.
%% keywords here, in the form: keyword \sep keyword

%Program Comprehension \sep
%Software exploration \sep
%Code Comprehension \sep
%Software maintenance \sep
%Software visualization \sep
%SAP development

%% PACS codes here, in the form: \PACS code \sep code

%% MSC codes here, in the form: \MSC code \sep code
%% or \MSC[2008] code \sep code (2000 is the default)

\end{keyword}

\end{frontmatter}

%% \linenumbers

%%%%%%%%%%%%%%%%%%%%%%%%%%%%%%%%%%%%%%%%%%%%%%%%%%%%%%
\section{Introduction}
\label{intro}
This document describes how Moose2Model extracts a SOMIX metamodel~\cite{r_SOMIX} from file. 
This extraction follows the principles which are described in~\cite{r_Metamodel_Preprint_2}. 

\section{Extract File System}
Directories and files of a file system are mapped as follows.

Directories are mapped to SOMIX.Grouping. Fields:
\begin{itemize}
\item name: The name of the directory without any conversion of cases.
\item uniqueName: All directories where the specified directory is contained separated by a slash.
\item technicalType: 'File'.
\item linkToEditor: A link to open the specified element in Visual Studio Code.
\end{itemize}

Files are mapped to SOMIX.Data. Fields:
\begin{itemize}
\item name: The name of the file without any conversion of cases.
\item uniqueName: All directories where the specified directory is contained separated by a slash. Followed by a slash and the name of the file.
\item technicalType: 'Directory'.
\item linkToEditor: A link to open the specified element in Visual Studio Code.
\end{itemize}

Files and folders are contained in folders. These is mapped to SOMIX.ParentChild.

\section{Extract JavaScript}
JavaScript is expected to be the coding in html pages. The extraction logic searches therefore for the \textless script\textgreater~tag in files which contain html. JavaScript coding is expected to be in the html file itself and in linked files which contain JavaScript code.


\section*{Declaration of Competing Interests} 
The author declares that he has no known competing financial interests or personal relationships that could have appeared to influence the work reported in this paper.



\section*{Acknowledgments}

This work was funded by CubeServ GmbH.
Many colleagues provided valuable support to make this project possible.
I thank
Patrick Michels for providing resources and support.



\begin{thebibliography}{00}

\bibitem{r_SOMIX}
Moose2Model, \url{https://github.com/Moose2Model/SOMIX} (accessed 12 September 2022).
\bibitem{r_Metamodel_Preprint_2}
Winkler, Rainer Wolfgang, A Software Metamodel Restricted to Key Aspects of a Software System for Developers and a Method to Keep Manually Drawn Diagrams Up-to-Date and Correct. Available at SSRN: https://ssrn.com/abstract=4049604 or http://dx.doi.org/10.2139/ssrn.4049604.


\end{thebibliography}
\end{document}

\endinput

